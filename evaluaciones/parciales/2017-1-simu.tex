% Author : Juan Garc\'ia Garland
% License: GPLv3

% This is a plain TeX file. Use the tex system to compile it.

\nopagenumbers
\parindent=0in

\font \sectionsize = cmr10 scaled \magstep 5
\font \titlesize = cmr10 scaled \magstep 2

\def\center#1{
\par
\begingroup
\leftskip=0pt plus 1fil
\rightskip=\leftskip
\parindent=0pt
\parfillskip=0pt
#1
\par
\endgroup
}

\def\title#1{{\vskip 10pt \titlesize \bf \center{#1}}}

\def\ej#1{ {\vskip 30pt \sectionsize \bf #1} }

\def\part#1{{\parindent=10pt \sectionsize \bf  \itemitem{#1)}}}

\def\ejsec#1#2{{\parindent=10pt \sectionsize \bf  \itemitem{#1)}}
               { #2}}

\def\mathbb#1{{\rm I\!#1}}


%Document begins
\title{Matem\'atica discreta y L\'ogica I - Ejemplo de parcial}

\center{San Jos\'e -- Abril de 2017}

No se permite consultar material.

No se permite utilizar calculadora.

Se disponen de 3 horas.

Justifique todas sus respuestas.

\ej{Ejercicio 1.}

Sea $A = \{a,b,c,d\}$ un conjunto y $R \in A \times A$ una la relaci\'on
sobre $A$, tal que:
$$R = \{(a,a),(a,b),(a,d),(b,a),(b,b),(b,c),(b,d)
,(c,b),(c,c),(d,a),(d,b),(d,d)\}.$$

\part{a}
Determinar una matriz de adyacencia para esta relaci\'on.
\part{b}
Determinar si la relaci\'on es reflexiva, sim\'etrica, y/o transitiva,
justificando seg\'un los criterios {\bf sobre la matriz}
vistos en el curso.



\ej{Ejercicio 2.}

Sea $(a_n)_{n \in \mathbb{N}}$.
Consideramos la ecuaci\'on:

$$a_n - 5a_{n-1} = -6a_{n-2} \hbox{ ~~ } (n \geq 2)$$

\part{a}
Hallar la soluci\'on general.

\part{b}
Hallar una soluci\'on particular sabiendo que $a_0 = 1$, y $ a_1 = 4$


\ej{Ejercicio 3.}

\part{a}
Escribir (en binario) la suma $10010_2 + 1011010_2$

\part{b}
Representar $584_{10}$ en base $2$.

\part{c}
Representar {\tt 0xCAFE} en base $2$.

\ej{Ejercicio 4.}

Sea $A$ un conjunto. Consideramos una operaci\'on cerrada en $A$.
$f : A \times A \rightarrow A$.
Decimos que $e \in A$ es el {\it elemento neutro} de la operaci\'on $f$ si
$\forall a \in A, f(a,e) = f(e,a) = a$.

\part{a}
Demostrar que si existe un neutro, este tiene que ser \'unico.

\part{b}
Supongamos que $A = \mathbb{N}$, y $f(m,n) = m+n-2$.
Hallar (si existe) el neutro para esta operaci\'on. En caso de que no exista,
justificar por qu\'e.

\eject

\bye


