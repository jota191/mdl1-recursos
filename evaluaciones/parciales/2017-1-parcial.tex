% Author : Juan Garc\'ia Garland
% License: GPLv3

% This is a plain TeX file. Use the tex system to compile it.

\nopagenumbers
\parindent=0in

\font \sectionsize = cmr10 scaled \magstep 5
\font \titlesize = cmr10 scaled \magstep 2

\def\center#1{
\par
\begingroup
\leftskip=0pt plus 1fil
\rightskip=\leftskip
\parindent=0pt
\parfillskip=0pt
#1
\par
\endgroup
}

\def\title#1{{\vskip 10pt \titlesize \bf \center{#1}}}

\def\ej#1{ {\vskip 30pt \sectionsize \bf #1} }

\def\part#1{{\parindent=10pt \sectionsize \bf  \itemitem{#1)}}}

\def\ejsec#1#2{{\parindent=10pt \sectionsize \bf  \itemitem{#1)}}
               { #2}}

\def\mathbb#1{{\rm I\!#1}}


%Document begins
\title{Matem\'atica discreta y L\'ogica I - Primer parcial}

\center{San Jos\'e -- 4 de mayo de 2017}

No se permite consultar material.

No se permite utilizar calculadora.

Se disponen de 3 horas.

Justifique todas sus respuestas.

\ej{Ejercicio 1.}

Sea $A = \{a,b,c,d,e\}$ un conjunto y $R \in A \times A$ una la relaci\'on
sobre $A$, tal que:
$$R = \{(a,a),(a,b),(a,c),(a,d),(a,e),
(b,b),(c,c),(c,e),(d,d),(d,e),(e,e) \}.$$

\part{a}
Determinar una matriz de adyacencia para esta relaci\'on.
\part{b}
Determinar si la relaci\'on es reflexiva, sim\'etrica, y/o transitiva,
justificando seg\'un los criterios {\bf sobre la matriz}
vistos en el curso.
\part{c}
Concluir que se trata de un \'orden parcial, realizar el diagrama de Hasse para
el mismo.
\part{d}
?`Es un ret\'iculo? {\bf Justifique.}


\ej{Ejercicio 2.}

Sea $(a_n)_{n \in \mathbb{N}}$.
Consideramos la ecuaci\'on:

$$a_n - 2a_{n-1} = -a_{n-2} \hbox{ ~~ } (n \geq 2)$$

\part{a}
Hallar la soluci\'on general.

\part{b}
Hallar una soluci\'on particular sabiendo que $a_0 = 1$, y $ a_1 = 4$


\ej{Ejercicio 3.}

\part{a}
Escribir (en binario) la suma $111010_2 + 101101_2$

\part{b}
Representar $614_{10}$ en base $2$.

\part{c}
Representar {\tt 0xBABA} en base $2$.


\ej{Ejercicio 4.}

Sean $A = \{1,2,5,6\}$, $B = \{1,3,4,5\}$, en donde $U = \mathbb{N}$


Definir:

\part{a} $(A \cup B)^{c}$

\part{b} $A \cap B$

\part{c} $A - B$

\hfill
\eject

\ej{Ejercicio 5.}

Resolver {\bf uno} de los siguientes problemas:


\part{O1}

Sea $A$ un conjunto. Consideramos una operaci\'on cerrada en $A$.
$f : A \times A \rightarrow A$.
Decimos que $e \in A$ es el {\it elemento neutro} de la operaci\'on $f$ si
$\forall a \in A, f(a,e) = f(e,a) = a$.

Demostrar que si existe un neutro, \'este tiene que ser \'unico.


\part{O2}
Consideramos la ecuaci\'on de diferencias:

$$ a_n - k a_{n-1} = 0$$

Sobre sucesiones de reales ($k \in \mathbb{R}$).

Demostrar que las sucesiones de la forma $a_n = C k^n$, con $C \in \mathbb{R}$
satisfacen la ecuaci\'on.

\hfill

\bye


