%Author: Juan Garc\'ia Garland
\documentclass[12pt,a4paper]{article}
\usepackage{amsmath}
\usepackage{amsfonts}


\ifdefined\accmod
\usepackage{helvet}
\renewcommand{\seriesdefault}{\bfdefault}
\renewcommand{\familydefault}{\sfdefault}  
\boldmath
\fi


\begin{document}

\section*{Ejercicio 1}
Demostrar que, para $n \in \mathbb{N}, n>1$ se cumple:
\begin{equation*}
  \sum_{i=1}^{n}{(2i-1)} = n^2
\end{equation*}

\section*{Ejercicio 2}
Sea el conjunto $A= \{ a,b,c,d,e\}$. Consideramos la relaci\'on
$R\subset A \times A$ definida por:

\begin{align*}
  R = \{&(c,c),(c,a),(c,b),(c,e),(c,d),(a,a),(a,d),(a,e),\\
     & (b,b),(b,d),(b,e),(e,e),(e,d),(d,d)
  \}
\end{align*}

\begin{enumerate}
\item
  Construir una matriz de adyacencia para la relaci\'on $R$.
\item
  Decidir si $R$ cumple o no las siguientes propiedades:
  \begin{itemize}
  \item Reflexividad
  \item Simetr\'ia
  \item Antisimetr\'ia
  \item Transitividad
  \end{itemize}
  En cada caso, justificar.
\item
  Es una relaci\'on de equivalencia? En caso afirmativo hallar
  $A/R$.
\item
  Es un orden parcial? En caso afirmativo, dibujar el diagrama de Hasse
  correspondiente.
\end{enumerate}

\newpage
\section*{Ejercicio 3}
Consideramos los conjuntos:
\begin{align*}
  A &= \{ n \in \mathbb{N} / n < 10\} \\
  B &= \{ 2,3,5 \}
\end{align*}
Calcular:
\begin{enumerate}
\item $A \cap B$
\item $A \cup B$
\item $A-B$
\item $A \times B$
\item $|\mathcal{P}(B)|$
\end{enumerate}

\section*{Ejercicio 4}

Considere $f: \mathbb{N}\times\mathbb{N}\rightarrow\mathbb{N}$ definida
como $f(x,y)= 2(x+y)$.
\begin{enumerate}
\item Es $f$ inyectiva?
\item Es $f$ sobreyectiva?
\item Es $f$ biyectiva?
  En cada caso, justifique.

\item Sea $g :\mathbb{N}\rightarrow\mathbb{N}\times\mathbb{N}$ tal que
  $f(x) = (x,x)$. Calcule $g \circ f$ y $f \circ g$.
\end{enumerate}

\section*{Ejercicio 5 (Te\'orico)}
Sea $f : A\times A\rightarrow A$ una operaci\'on sobre $A$.
Definir elemento neutro para $A$.

\end{document}
