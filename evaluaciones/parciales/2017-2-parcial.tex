% Author : Juan Garc\'ia Garland
% License: GPLv3

% This is a plain TeX file. Use the tex system to compile it.

\nopagenumbers
\parindent=0in

\font \sectionsize = cmr12 scaled \magstep 5
\font \titlesize = cmr12 scaled \magstep 2

\font\big=cmr10 at 11pt

\def\center#1{
\par
\begingroup
\leftskip=0pt plus 1fil
\rightskip=\leftskip
\parindent=0pt
\parfillskip=0pt
#1
\par
\endgroup
}

\def\title#1{{\vskip 10pt \titlesize \bf \center{#1}}}

\def\ej#1{ {\vskip 30pt \sectionsize \bf #1} }

\def\part#1{{\parindent=10pt \sectionsize \bf  \itemitem{#1)}}}

\def\ejsec#1#2{{\parindent=10pt \sectionsize \bf  \itemitem{#1)}}
               { #2}}

\def\mathbb#1{{\rm I\!#1}}


%Document begins
\title{Matem\'atica Discreta y L\'ogica I - Segundo Parcial}

\center{San Jos\'e -- Lunes 26 de junio de 2017}

\big

\vfill

No se permite consultar material.

Se disponen de 3 horas.

Justifique todas sus respuestas.

\ej{Ejercicio 1.}

Considere el alfabeto $ \Sigma = \{ {\tt a,b,c }\}$.

\par{a) Defina inductivamente el lenguaje $L  \subset \Sigma^*$ tal que
$L = \{ {\tt \epsilon, ac, baca, bbacaa, bbbacaaa, bbbbacaaaa, ... } \}$.

\par{b)} Enuncie el principio de inducci\'on primitiva (PIP) para $L$.

\par{c)} Defina siguiendo el esquema de recursi\'on primitiva (ERP),
las funciones $cant_a, cant_b : L \rightarrow \mathbb{N}$
que cuentan la cantidad de ocurrencias de los s\'\i mbolos {\tt a} y {\tt b}
respectivamente, por ejemplo tenemos que $cant_a ({\tt baca}) = 2$,
$cant_b ({\tt bbbacaaa}) = 3$.

\par{d)} Demuestre, utilizando el PIP para $L$, que para toda palabra
$w \in L$, se cumple que $cant_a(w) = cant_b(w) + 1$.

\par{e)} Demuestre que ${\tt bbacaa} \in L$.
Justifique (de manera informal) que ${\tt bbaacaa} \notin L$.




\ej{Ejercicio 2.}

Demostrar o refutar, utilizando tablas de verdad las siguientes afirmaciones:

\par{a)} $
          \models (p_0 \rightarrow p_1) \leftrightarrow (\neg p_1 \rightarrow
          \neg p_0)
         $

\par{b)} $\models (p_0 \vee p_1 \vee \neg p_2) \rightarrow
          \neg (\neg p_0 \wedge \neg p_1 \wedge p_2)$


\ej{Ejercicio 3.}

Demostrar:

\par{a)} $\alpha \leftrightarrow \beta, \alpha \models \beta$

\par{b)} $\models (\alpha \leftrightarrow \beta)
          \leftrightarrow (\neg \beta \leftrightarrow \neg \alpha)$


\ej{Ejercicio 4.}

Seleccione 3 de los siguientes teoremas, y construya
derivaciones que los demuestren.
No se admiten en ning\'un caso justificaciones sem\'anticas.

\par{a)} $\alpha \vdash \neg ( \neg \alpha \wedge \neg \beta)$

\par{b)} $\vdash \alpha \wedge \beta \rightarrow \beta \wedge \alpha$

\par{c)} $\vdash \neg \alpha \rightarrow (\alpha \rightarrow \bot )$

\par{d)} $\vdash ( \alpha \leftrightarrow \beta )
            \leftrightarrow
            (\alpha \rightarrow \beta) \wedge (\beta \rightarrow \alpha)
         $

\hfill

\bye


