%Author: Juan Garc\'ia Garland
\ifdefined\accmod
\documentclass[14pt]{extarticle}
\usepackage{geometry}
%\usepackage[a4paper,margin=1in,landscape]{geometry}

\else
\documentclass[12pt,a4paper]{article}
\fi
\usepackage{amsmath}
\usepackage{amsfonts}


\usepackage{fancyhdr}
\usepackage{lastpage}

\pagestyle{fancy}
\fancyhf{}
\rhead{Primer Parcial}
\lhead{Matem\'atica Discreta y L\'ogica I}
\rfoot{P\'agina \thepage \: de \pageref{LastPage}}


%environment para exos
\newtheorem {Ejercicio}{Ejercicio}
\newenvironment{ejercicio}{\begin{Ejercicio} \rm}{\rm\end{Ejercicio}}


% para compilar bold y helvet, para mayor accesibilidad
\ifdefined\accmod
\usepackage{helvet}
\renewcommand{\seriesdefault}{\bfdefault}
\renewcommand{\familydefault}{\sfdefault}  
\boldmath
\fi


\begin{document}

\noindent
Tiempo: 2 horas.\\
No se permite consultar material ni calculadora.\\
Justifique todas sus respuestas.

\begin{ejercicio} \hfill \break
  Sean $A = \{ n \in \mathbb{N} \:/\:  n\:\mbox{es par y}\: n<7\}$,
  $B=\{4,6,8,10\}$. Se pide:
  \begin{enumerate}
  \item Escribir una definici\'on de $B$ por comprensi\'on.
  \item Hallar $A\cup B$, $A\cap B$, $A-B$, $B-A$.
  \item Calcular $|\mathcal{P}(B)|$.
  \item Calcular cu\'antas funciones hay con
    dominio $A$ y codominio $B$.
  \end{enumerate}
\end{ejercicio}

\begin{ejercicio}\hfill \break
  Demostrar que, para $n\in\mathbb{N}$:
  $$\sum_{i=1}^{n}{(2i-1)} = n^2$$
\end{ejercicio}


\ifdefined\accmod
\newpage
\fi

\begin{ejercicio}\hfill \break
  Sea $A=\{1,2,3,4,5\}$ un conjunto y
  $$R = \{(1,1),(1,3),(3,1),(3,3),(2,4),(4,2),(2,2),(4,4),(5,5)\}$$
  una relaci\'on sobre $A$.
  \begin{enumerate}
  \item Hallar una matriz de adyacencia para la relaci\'on $R$.
  \item Decidir si la relaci\'on $R$ es:
    \begin{enumerate}
    \item Reflexiva.
    \item Sim\'etrica.
    \item Antisim\'etrica.
    \item Transitiva.
    \end{enumerate}
  \item
    Es una relaci\'on de equivalencia? En caso afirmativo hallar
    $A/R$.
  \item
    Es un orden parcial? En caso afirmativo,
    dibujar el diagrama de Hasse correspondiente.
  \end{enumerate}
\end{ejercicio}



\begin{ejercicio}\hfill \break

  Sean $f:\mathbb{N}\rightarrow\mathbb{N}$
  y
  $g:\mathbb{N}^2\rightarrow\mathbb{N}$.
  
  Donde $f(n)=2n$, y $g(m,n)=m+n$.



  \begin{enumerate}
  \item
    Calcular $f \circ g$.
  \item
    Puede calcular $g \circ f$? Justifique.
  \end{enumerate}

\end{ejercicio}
\end{document}
