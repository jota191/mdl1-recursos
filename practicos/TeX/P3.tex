\documentclass[12pt,a4paper]{article}

\usepackage[spanish]{babel}
\usepackage[latin1]{inputenc}
\usepackage{graphicx}

\hoffset = -.5in
\marginparwidth = 0cm
\evensidemargin = 0cm
\marginparpush = 0cm
\voffset = -1in



\textwidth 17cm \textheight 25cm

\usepackage{amsmath}
\usepackage{amsfonts}

\renewcommand{\theenumi}{\Alph{enumi}}

\begin{document}
\pagestyle{empty}

\newtheorem {Ejercicio}{Ejercicio}
\newenvironment{ejercicio}{\begin{Ejercicio} \rm}{\rm\end{Ejercicio}}

\hrule
\bigskip
\noindent Matem�tica Discreta y L�gica 1\\
\noindent Sede San Jos�

\smallskip


\begin{center}
{\large\sc  Pr�ctico 3: Funciones}
\end{center}
\medskip
\hrule

\bigskip

\begin{ejercicio}
Determinar cuales de las relaciones definidas en el primer ejercicio del pr�ctico anterior son funciones.
\end{ejercicio}

\begin{ejercicio}
Determinar cu�l(es) de las siguientes propiedades verifican las funciones que se definen abajo
\begin{itemize}
\item
inyectiva,
\item
sobreyectiva,
\item
biyectiva.
\end{itemize}

\begin{enumerate}
\item
$f: \mathbb{Z} \to \mathbb{Z}$ definida como $f(z)=z-1$.
\item
$f: \mathbb{N} \times \mathbb{N} \to \mathbb{N}$ definida como $f(x,y)=x$.
\item
$f: \mathbb{N} \times \mathbb{N} \to \mathbb{N}$ definida como $f(x,y)=x+y$.
\item
$f: \mathbb{N} \times \mathbb{N} \to \mathbb{N}$ definida como $f(x,y)=xy$.


\end{enumerate}

\end{ejercicio}


\begin{ejercicio}
Se consideran conjuntos $A$ y $B$ con $n$ y $m$ elementos respectivamente.
\begin{enumerate}
\item
�Cu�ntas funciones de $A$ en $B$ se pueden definir?
\item
  �Qu� relaci�n deben cumplir $m$ y $n$ para poder definir una funci�n inyectiva
  de $A$ en $B$?\\
�Cu�ntas funciones inyectivas de $A$ en $B$ se pueden definir?
\item
  �Qu� relaci�n deben cumplir $m$ y $n$ para poder definir una funci�n
  sobreyectiva de $A$ en $B$? �Qu� relaci�n deben cumplir $m$ y $n$ para
  poder definir una funci�n biyectiva de $A$ en $B$?\\
�Cu�ntas funciones biyectivas de $A$ en $B$ se pueden definir?
\end{enumerate}
\end{ejercicio}

\begin{ejercicio}
  Probar que la funci�n de $f:\mathbb{N}\to P$,
  donde $P$ es el conjunto de los n�meros pares
  (es decir, $\{2n: n \in \mathbb{N}\}$) definida como $f(n)=2n$ es biyectiva.
  �Qu� podr�a decirse de la cantidad de elementos de estos conjuntos?
\end{ejercicio}

\begin{ejercicio}
  Para los siguientes pares de funciones $f:\mathbb{Z} \rightarrow \mathbb{Z}$
  calcular $f \circ g$ y $g \circ f$. 
  \begin{enumerate}
  \item $f(x) = x$, $g(x) = x$
  \item $f(x) = x^2$, $g(x) = 4x$
  \item $f(x) = 2x + 5$, $g(x) = 3x - 1$
  \item $f(x) = 2x - 1$, $g(x) = x^{3} - x^{2} - 4$
  \end{enumerate}
\end{ejercicio}


\begin{ejercicio}
  Consideramos las operaciones $\oplus$ y $\otimes$ en $ \{ 0,1\}$, definidas como:

  $$a \oplus b = \left \{ \begin{matrix} 0 & \mbox{si } a = b = 0
    \\ 1 & \mbox{en otro caso} \end{matrix}\right. , \hspace{1cm}
  a \otimes b = \left \{ \begin{matrix} 1 & \mbox{si } a = b = 1
    \\  0 & \mbox{en otro caso} \end{matrix}\right.$$

  Determinar si tienen neutro y encontrarlo.
\end{ejercicio}




\end{document}
