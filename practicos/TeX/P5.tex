\documentclass[12pt,a4paper]{article}

\usepackage[spanish]{babel}
\usepackage[latin1]{inputenc}
\usepackage{graphicx}
\usepackage{hyperref}
\usepackage{enumitem}

\hoffset = -.5in
\marginparwidth = 0cm
\evensidemargin = 0cm
\marginparpush = 0cm
\voffset = -1in



\textwidth 17cm \textheight 25cm

\usepackage{amsmath}
\usepackage{amsfonts}

\newcommand{\N}{\mathbb{N}}

\newcounter{cuent}
\newcommand{\proba}[1]{\stepcounter{cuent}{\alph{cuent})\quad}
\displaystyle#1\qquad}
\newcommand{\cuento}{\setcounter{cuent}{0}}


\begin{document}
\pagestyle{empty}

\newtheorem {Ejercicio}{Ejercicio}
\newenvironment{ejercicio}{\begin{Ejercicio} \rm}{\rm\end{Ejercicio}}

\hrule
\bigskip
\noindent Matem�tica Discreta y L�gica I\\
\noindent Sede San Jos�

\smallskip
\begin{center}
{\large\sc  Pr�ctico 5: Sucesiones definidas por Recurrencia}
\end{center}
\medskip
\hrule

\bigskip


\begin{ejercicio}
  Consideramos las siguientes sucesiones definidas por recurrencia:
  
  $$\proba{\left\{\begin{matrix} a_0 =& 6 &\\
    a_n =& 4 a_{n-1} &(n\geq 1)\end{matrix}
    \right.  }
  \proba{\left\{\begin{matrix} b_0 =& 1 &\\
    b_{n+1} =& 2 b_n &(n \geq 0)\end{matrix}
    \right. }
  \proba{ \left\{\begin{matrix} c_0 =& 2 &\\
    c_n =& (-3) c_{n-1} &(n> 0)\end{matrix}
    \right. }  $$
  Hallar una expresi\'on expl\'icita en cada caso,
  demostrar que es correcta por inducci\'on.
\end{ejercicio}



\begin{ejercicio}
  Encuentre la soluci\'on general de las siguientes relaciones de recurrencia,
  y las soluciones particulares sabiendo que $a_0=1$ :
  \begin{enumerate}[label=\alph* )]
  \item
    $3 a_{n+1} - 4 a_n = 0 (n \geq 0)$
  \item
    $2 a_{n+1} - 3 a_n = 0 (n \geq 0)$
  \item
    $2 a_{n+1} -  \frac{3 a_n}{2} = 0 (n \geq 0)$
  \end{enumerate}
\end{ejercicio}


\begin{ejercicio}
  Consideramos las siguientes sucesiones definidas por recurrencia:
  \cuento
  $$
  \proba{\left\{\begin{matrix} a_0 =& 2 &\\
    a_n =& 3 + a_{n-1} &(n\geq 1)\end{matrix}
    \right.  }
  \proba{\left\{\begin{matrix} b_0 =& 3 &\\
    b_{n+1} =& b_n -2 &(n \geq 0 )\end{matrix}
    \right. } $$

  Hallar una expresi\'on expl\'icita en cada caso,
  demostrar que es correcta por inducci\'on.
\end{ejercicio}


\begin{ejercicio}
  Generalizar el ejercicio anterior, dada:
  $$
  \left\{\begin{matrix} a_0 =& C &\\
    a_n =& a_{n-1} + K &(n\geq 1)\end{matrix}
    \right. $$

    Hallar una expresi\'on expl\'icita para $a_n$. Demostrar por inducci\'on
    que es correcta.
\end{ejercicio}


\end{document}
