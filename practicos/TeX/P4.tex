\documentclass[12pt,a4paper]{article}

\usepackage[spanish]{babel}
\usepackage[latin1]{inputenc}
\usepackage{graphicx}
\usepackage{hyperref}
\usepackage{enumitem}

\hoffset = -.5in
\marginparwidth = 0cm
\evensidemargin = 0cm
\marginparpush = 0cm
\voffset = -1in



\textwidth 17cm \textheight 25cm

\usepackage{amsmath}
\usepackage{amsfonts}

\newcommand{\N}{\mathbb{N}}

\begin{document}
\pagestyle{empty}

\newtheorem {Ejercicio}{Ejercicio}
\newenvironment{ejercicio}{\begin{Ejercicio} \rm}{\rm\end{Ejercicio}}

\hrule
\bigskip
\noindent Matem�tica Discreta y L�gica I\\
\noindent Sede San Jos�

\smallskip


\begin{center}
{\large\sc  Pr�ctico 4: Inducci�n}
\end{center}
\medskip
\hrule

\bigskip

\begin{ejercicio}

  \noindent
  Se consideran los siguientes teoremas sobre los naturales:
\begin{enumerate}[label= \alph*)]
\item
  $\forall n \in \N, n^3 - n = \dot{3}$
\item
  $\forall n \in \N, \sum_{i=0}^{n}{i} = \frac{n(n+1)}{2}$
\item
  $\forall n \in \N, \sum_{i=0}^{n}{i^2} = \frac{n(n+1)(2n+1)}{6}$
\item
  $\forall n \in \N, \sum_{i=0}^{n}{i^3} = \frac{1}{4} n^2 (n+1)^2$
\item
  $\forall n \geq 2, f_n^2 = f_{n-1} f_{n+1} + (-1)^{n}$
  En donde $(f_i)_{i \in \N}$ es la sucesi\'on de Fibonacci.
\item
  $\forall n \geq 4, n!>2^n$
\item
  $\forall n \geq 10, n^3<2^n$
\end{enumerate}

Se pide:
\begin{enumerate}[label= \alph*)]
\item Plantear en cada caso el esquema de la prueba por inducci�n, es decir:
  el paso base, el paso inductivo,
  y dentro de �ste �ltimo la hip�tesis y tesis, como se vi� en clase.
\item
  Realizar la prueba por inducci�n (es decir,
  chequear que el paso base se satisface
  y demostrar la tesis inductiva a partir de la hip�tesis)
\end{enumerate}

\end{ejercicio}

\begin{ejercicio}
  Calcular los primeros 5 t�rminos de las siguientes funciones de tipo
  $\N \rightarrow \N$ definidas por recursi�n:
  
  \begin{enumerate}
    \item
      \[ f(n) =
      \begin{cases} 
        1        & \mbox{si } n = 0 \\
        2 f(n-1) & \mbox{si } n \geq 1 
      \end{cases}
      \]

    \item
      \[ g(n) =
      \begin{cases} 
        2        & \mbox{si } n = 0 \\
        4        & \mbox{si } n = 1 \\
        2 f(n-1) + f(n-2) & \mbox{si } n \geq 2 
      \end{cases}
      \]

    \item
      \[ h(n) =
      \begin{cases} 
        1        & \mbox{si } n = 0 \\
        -1       & \mbox{si } n = 1 \\
        -2 f(n-1) + f(n-2) & \mbox{si } n > 1 
      \end{cases}
      \]
  \end{enumerate}
\end{ejercicio}

\begin{ejercicio}
  Demostrar por inducci�n que para todo natural $n$, $g(n)$ es par
  (siendo $g(n)$ la funci�n definida en el ejercicio anterior).
\end{ejercicio}

\newpage 
\begin{ejercicio}
  Definir por recursi�n las siguientes funciones:
  \begin{enumerate}[label= \alph*)]
  \item $f(n) = 2n$
  \item La funci�n $fib(n)$ que retorna el en�simo n�mero de fibonacci.
    \footnote{\href{https://en.wikipedia.org/wiki/Fibonacci\_number}
      {https://en.wikipedia.org/wiki/Fibonacci\_number}}
  \item
    $n!$
  \item
    $f(n)=2^n$
  \end{enumerate}
\end{ejercicio}

\end{document}
