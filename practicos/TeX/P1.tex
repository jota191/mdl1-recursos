\documentclass[12pt,a4paper]{article}

\usepackage[spanish]{babel}

\usepackage[latin1]{inputenc}
\usepackage{graphicx}
\usepackage{multicol}

\hoffset = -.5in

\marginparwidth = 0cm

\evensidemargin = 0cm

\marginparpush = 0cm



\voffset = -1in



\textwidth 17cm \textheight 25cm





%\usepackage{amssymb}

\usepackage{amsmath}

\usepackage{amsfonts}









\renewcommand{\theenumi}{\Alph{enumi}}



\begin{document}

\pagestyle{empty}





\newtheorem {Ejercicio}{Ejercicio}
\newenvironment{ejercicio}{\begin{Ejercicio} \rm}{\rm\end{Ejercicio}}



\hrule
\bigskip
\noindent Matem�tica Discreta y L�gica 1\\
\noindent Sede Buceo

\smallskip


\begin{center}
{\large\sc  Pr�ctico 1: Conjuntos}
\end{center}
\medskip
\hrule

\begin{enumerate}
\item[1.]
Realizar un dibujo que ilustre las operaciones: uni�n, intersecci�n, complemento y diferencia.

\item[2.]
Describir por extensi�n (si es posible) los siguientes conjuntos:

\setlength{\columnsep} {-2.5cm}
\begin{multicols}{2}

\begin{enumerate}

\item $D= A \cup B $,
\item $E= A \cap C $,
\item $D \setminus C$,
\item $ B^c$ en el universo $\mathbb{N}$,
\item $A \times C$,
\item $C \times A$,
\item $\{n \in \mathbb{N}: n<0\}$.


\end{enumerate}
\end{multicols}

siendo
\[
A= \{2,4,6,8,10\}, \qquad B = \{1,3,5,7,9\}, \qquad C= \{3, 6, 9\}.
\]

\item[3.]
Describir por comprensi�n los siguientes conjuntos:

\setlength{\columnsep} {-2.5cm}
\begin{multicols}{2}

\begin{enumerate}

\item $A= \{2,4,6,8,10\} $,
\item$ B = \{1,3,5,7,9\}$,
\item $C= \{3, 6, 9\}$,
\item $D= A \cup B $,
\item $E= A \cap C $,
\item $D \setminus C$,
\item $ B^c$ en el universo $\mathbb{N}$.
\end{enumerate}
\end{multicols}

\item[4.]
Sean $\mathcal{U} = \{n \in \mathbb{N}\colon n \leq 10\}, A = \{1,3,5,7,9\}, B = \{4,5,6\} $ y $ C= \{2, 4, 6, 8\}$.\\
Describir por extensi�n y por comprensi�n los siguientes conjuntos:
\setlength{\columnsep} {-2.5cm}
\begin{multicols}{3}

\begin{enumerate}
\item $A \cup \emptyset$,
\item $A \cap \emptyset $,
\item $A \cap (B \cup C)$,
\item $B^c \cap (C \setminus A)$,
\item $(A \cup B)^c \cup C$,
\item $(A \cup B)\setminus (C \setminus B)$.
\end{enumerate}
\end{multicols}
\item[5.]
Establecer si existe relaci�n de 
\begin{itemize}
\item (no) pertenencia,
\item (no) inclusi�n,
\item igualdad,
\end{itemize}
entre los siguientes pares de objetos:

\setlength{\columnsep} {-4.5cm}
\begin{multicols}{2}

\begin{enumerate}
\item $\{1,2,3,0\}$ y $0$, 
\item $\{1,2,3,0\}$ y $\emptyset$,
\item $\{1,2,3,0\}$ y $5$,
\item $\{1,2,3,0\}$ y $\{5\}$,
\item $\{1,2,3,0\}$ y $\{5,0\}$,
\item $\{(1,1), (2,3), (1,5), (0,0), (5,0)\}$ y $(5,0)$,
\item $\{(1,1), (2,3), (1,5), (0,0), (5,0)\}$ y $\{(5,0)\}$,
\item $\{0\}$ y $\emptyset$,
\item $\{(1,1), (2,3)\}$ y $\{(1,1), (3,2)\}$.


\end{enumerate}
\end{multicols}
\end{enumerate}



\end{document}
